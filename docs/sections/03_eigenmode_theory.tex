% 03_eigenmode_theory.tex
\section{Eigenmode Theory and Dispersion Relation}

\subsection{LP Mode Solutions}

The scalar Helmholtz equation in the two regions admits closed-form solutions in terms of Bessel functions. For the fundamental LP$_{01}$ mode (azimuthal number $m = 0$), we seek radially symmetric solutions that remain regular at the origin and decay exponentially as $r \to \infty$.

\subsubsection{Core Region Solution}

In the core region $(r < a)$, the equation becomes:
\[
\frac{1}{r}\dv{r}\left( r\dv{\Psi}{r} \right)
+ \left( k_0^2 n_1^2 - \beta^2 \right) \Psi(r) = 0.
\]

Define the core eigenvalue parameter:
\[
u = a\sqrt{k_0^2 n_1^2 - \beta^2},
\]
so that the equation in dimensionless form becomes:
\[
\frac{u^2}{r^2}\dv{u}\left( u\dv{\Psi}{u} \right) + \Psi(u) = 0.
\]

This is the Bessel equation of order zero, with general solution:
\[
\Psi_{\text{core}}(r) = A \, J_0\left( \frac{ur}{a} \right),
\]
where $J_0$ is the Bessel function of the first kind and order zero. The constant $A$ is the amplitude, and we retain only $J_0$ (not $Y_0$) because the solution must be regular at $r = 0$.

\subsubsection{Cladding Region Solution}

In the cladding region $(r > a)$, the equation is:
\[
\frac{1}{r}\dv{r}\left( r\dv{\Psi}{r} \right)
+ \left( k_0^2 n_2^2 - \beta^2 \right) \Psi(r) = 0.
\]

Define the cladding eigenvalue parameter:
\[
w = a\sqrt{\beta^2 - k_0^2 n_2^2},
\]
which is real and positive for guided modes (since $\beta$ lies between $k_0 n_2$ and $k_0 n_1$). The equation becomes:
\[
\frac{w^2}{r^2}\dv{w}\left( w\dv{\Psi}{w} \right) - \Psi(w) = 0,
\]
which is the modified Bessel equation of order zero. The general solution is:
\[
\Psi_{\text{cladding}}(r) = C \, K_0\left( \frac{wr}{a} \right),
\]
where $K_0$ is the modified Bessel function of the second kind. We retain only $K_0$ (not $I_0$) because the solution must decay exponentially as $r \to \infty$.

\subsection{Normalized Transverse Parameters}

The dimensionless parameters $u$ and $w$ defined above are normalized transverse parameters that characterize the mode confinement. They satisfy the important identity:
\[
u^2 + w^2 = a^2 \left[ k_0^2(n_1^2 - n_2^2) \right] = V^2,
\]
where $V$ is the fiber V-number:
\[
V = k_0 a \sqrt{n_1^2 - n_2^2} = a \frac{2\pi}{\lambda_0} \sqrt{n_1^2 - n_2^2}.
\]

The V-number is the fundamental dimensionless parameter controlling fiber behavior:
\begin{itemize}
  \item \textbf{Single-mode regime:} $V < 2.405$ (only LP$_{01}$ mode propagates)
  \item \textbf{Few-mode regime:} $2.405 < V < 3.832$ (LP$_{01}$ and LP$_{11}$ modes)
  \item \textbf{Multimode regime:} $V \gg 1$ (many modes)
\end{itemize}

For a given $V$, once $u$ is determined from the dispersion relation, $w$ follows from $w = \sqrt{V^2 - u^2}$.

\subsection{Dispersion Relation}

Continuity of the field and its derivative at the core-cladding interface $r = a$ yields two equations. For the fundamental mode ($m = 0$), these reduce to a single characteristic equation:
\[
\frac{J_0(u)}{u J_1(u)} = -\frac{K_0(w)}{w K_1(w)}.
\]

This transcendental equation relates $u$ and $w$ through the fiber parameters $(V, n_1, n_2)$. Using the constraint $u^2 + w^2 = V^2$, we can write the dispersion relation as a single equation in $u$:
\[
\frac{J_0(u)}{u J_1(u)} + \frac{K_0(\sqrt{V^2 - u^2})}{\sqrt{V^2 - u^2} \, K_1(\sqrt{V^2 - u^2})} = 0.
\]

For a given wavelength $\lambda_0$ and fiber geometry $(n_1, n_2, a)$, the V-number is fixed, and solving this equation yields the corresponding value(s) of $u$, and hence $\beta$.

Key properties of the dispersion relation:
\begin{enumerate}
  \item \textbf{Existence:} A solution exists for all $V > 0$. The fundamental LP$_{01}$ mode never has a cutoff.
  \item \textbf{Uniqueness:} For the LP$_{01}$ mode, the solution $u(V)$ is unique and monotonically increasing with $V$.
  \item \textbf{Limits:} As $V \to 0$, $u \to 0$ (low-frequency limit); as $V \to \infty$, $u \to V$ (high-frequency limit).
\end{enumerate}

\subsection{Effective Index}

The propagation constant $\beta$ is related to the effective refractive index by:
\[
\beta = k_0 n_{\text{eff}},
\]
where $n_{\text{eff}}$ satisfies:
\[
n_2 < n_{\text{eff}} < n_1.
\]

From the definition of $u$, we can express:
\[
u^2 = a^2 k_0^2 (n_1^2 - n_{\text{eff}}^2),
\]
giving:
\[
n_{\text{eff}} = \frac{1}{k_0}\sqrt{k_0^2 n_1^2 - \frac{u^2}{a^2}}.
\]

Similarly, from the definition of $w$:
\[
n_{\text{eff}} = \frac{1}{k_0}\sqrt{k_0^2 n_2^2 + \frac{w^2}{a^2}}.
\]

The effective index is a key quantity because it determines:
\begin{itemize}
  \item \textbf{Phase velocity:} $v_p = c/n_{\text{eff}}$
  \item \textbf{Group velocity:} $v_g = c/n_g$, where $n_g = n_{\text{eff}} - \lambda_0 \dv{n_{\text{eff}}}{\lambda_0}$
  \item \textbf{Modal dispersion:} $D = -\lambda_0 \dv[2]{n_{\text{eff}}}{(\lambda_0)}$
  \item \textbf{Sensitivity to perturbations:} Changes in $n_1$, $n_2$, or $a$ directly affect $n_{\text{eff}}$, which is the focus of perturbation analysis
\end{itemize}

Numerically, $n_{\text{eff}}(\lambda_0)$ is computed by:
\begin{enumerate}
  \item Calculating $V = k_0 a \sqrt{n_1^2 - n_2^2}$ for the given $\lambda_0$
  \item Solving the transcendental dispersion relation for $u$ using numerical root-finding
  \item Computing $n_{\text{eff}}$ from either of the two expressions above
\end{enumerate}

This dispersion curve $n_{\text{eff}}(\lambda_0)$ is fundamental to understanding fiber behavior and forms the baseline for perturbation analysis in subsequent sections.

\begin{figure}[h!]
  \centering
  \includegraphics[width=0.9\textwidth]{../figures/02_mode_profile.pdf}
  \caption{%
    \textbf{LP$_{01}$ mode profile and power distribution.}
    Left: Radial field amplitude $|\Psi(r)|$ from core ($r < a$, blue) to cladding ($r > a$, red), showing Bessel $J_0$ oscillation in core and exponential $K_0$ decay in cladding.
    Right: Power density $|\Psi(r)|^2$ on logarithmic scale, illustrating the $\approx 75\%$ core confinement and exponential evanescent penetration in the cladding.
    The discontinuity in curvature at $r=a$ corresponds to the refractive index step.
  }
  \label{fig:mode_profile}
\end{figure}
