% 08_results.tex
\section{Results and Analysis}

\subsection{Effective Index Dispersion}

The effective index $n_{\text{eff}}(\lambda_0)$ as a function of wavelength is the fundamental characteristic of a guided mode. For a typical single-mode fiber with parameters:
\begin{itemize}
  \item Core radius: $a = 4.0 \, \mu\text{m}$
  \item Core index: $n_1 = 1.4504$ (germanosilicate glass)
  \item Cladding index: $n_2 = 1.4447$ (pure silica)
  \item Index contrast: $\Delta = (n_1 - n_2)/n_1 \approx 0.0039$ (0.39\%)
\end{itemize}

the dispersion curve exhibits characteristic behavior across the telecommunications wavelength range ($1.3$--$1.6 \, \mu\text{m}$):

\begin{enumerate}
  \item \textbf{Monotonic decrease:} $n_{\text{eff}}$ decreases with increasing wavelength, reflecting reduced confinement at longer wavelengths as the mode extends further into the cladding.
  
  \item \textbf{Bounded values:} The effective index satisfies $n_2 < n_{\text{eff}} < n_1$ for all guided modes, approaching $n_1$ at short wavelengths (strong confinement) and approaching $n_2$ near cutoff.
  
  \item \textbf{Cutoff behavior:} The fundamental LP$_{01}$ mode has no cutoff wavelength (it propagates for all $V > 0$). As $\lambda_0 \to \infty$, $V \to 0$ and $n_{\text{eff}} \to n_2$.
  
  \item \textbf{Operating point:} At $\lambda_0 = 1.55 \, \mu\text{m}$ (common telecommunications wavelength), typical values are $V \approx 2.2$ and $n_{\text{eff}} \approx 1.4470$, indicating moderate confinement with approximately 75--80\% of the mode power in the core.
\end{enumerate}

The effective index dispersion directly determines the group velocity dispersion (GVD) and chromatic dispersion of the fiber, which are critical parameters for telecommunications and sensing applications.

\begin{figure}[h!]
  \centering
  \includegraphics[width=0.9\textwidth]{../figures/03_dispersion_curve.pdf}
  \caption{%
    \textbf{Effective index dispersion and V-parameter.}
    Left: Effective index $n_{\text{eff}}(\lambda)$ versus wavelength, showing monotonic decrease characteristic of normal material dispersion.
    The curve spans $n_2 < n_{\text{eff}} < n_1$ as required by the eigenvalue problem, with values transitioning from strong confinement (short $\lambda$) to weak confinement (long $\lambda$).
    Right: Normalized frequency $V(\lambda)$ versus wavelength, showing V-parameter scaling and confirming the guided-mode cutoff at $V = 2.405$ for higher-order modes.
    All 44 computed points fall within the single-mode regime.
  }
  \label{fig:dispersion_curve}
\end{figure}

\subsection{LP\textsubscript{01} Mode Profile}

The radial intensity distribution $|\Psi(r)|^2$ of the fundamental mode reveals the spatial confinement characteristics:

\begin{itemize}
  \item \textbf{Core behavior:} In the core ($r < a$), the field follows $J_0(ur/a)$, exhibiting a maximum at the fiber center and decreasing monotonically toward the boundary. For typical V-numbers ($V \approx 2$--$2.4$), the profile is nearly Gaussian near the center.
  
  \item \textbf{Core-cladding interface:} At $r = a$, the field and its derivative are continuous, but the curvature (second derivative) exhibits a discontinuity proportional to the index step $n_1^2 - n_2^2$.
  
  \item \textbf{Cladding decay:} In the cladding ($r > a$), the field decays exponentially as $K_0(wr/a)$. The decay length is approximately $a/w$, which for $V \approx 2.2$ gives $w \approx 1.5$, corresponding to a penetration depth of about $0.67a$ (2.7 $\mu$m for $a = 4$ $\mu$m).
  
  \item \textbf{Power confinement:} The fraction of modal power in the core is:
  \[
  P_{\text{core}} = \frac{\int_0^a |\Psi(r)|^2 2\pi r \, dr}{\int_0^\infty |\Psi(r)|^2 2\pi r \, dr}.
  \]
  For typical single-mode fibers at $\lambda_0 = 1.55 \, \mu$m, $P_{\text{core}} \approx 0.75$--$0.80$, meaning that 20--25\% of the power propagates in the cladding as an evanescent field.
\end{itemize}

The mode profile determines the spatial sensitivity to perturbations: regions with high $|\Psi(r)|^2$ contribute more strongly to the perturbation integral, making core perturbations significantly more influential than cladding perturbations.

\subsection{Sensitivity Maps}

\subsubsection{Index Perturbation Sensitivity}

For a uniform index perturbation throughout the core ($\Delta n^2(r) = \Delta n_0^2$ for $r < a$), the effective index shift is:
\[
\Delta n_{\text{eff}} = \frac{1}{2n_{\text{eff}}} \Delta n_0^2 P_{\text{core}}.
\]

Representative results for the fiber parameters above at $\lambda_0 = 1.55 \, \mu$m:
\begin{itemize}
  \item For $\Delta n_0 = +0.001$ (0.07\% increase in core index): $\Delta n_{\text{eff}} \approx +7.7 \times 10^{-4}$ (using $P_{\text{core}} \approx 0.77$)
  \item For $\Delta n_0 = -0.001$: $\Delta n_{\text{eff}} \approx -7.7 \times 10^{-4}$
  \item Sensitivity coefficient: $\partial n_{\text{eff}}/\partial n_1 \approx 0.77$ (power confinement factor)
\end{itemize}

This linear relationship holds for $|\Delta n_0| \lesssim 0.01$. Beyond this range, second-order effects become significant and the perturbative approximation degrades.

For a localized perturbation in a thin shell at the core-cladding boundary ($a - \delta < r < a + \delta$ with $\delta = 0.5 \, \mu$m), the sensitivity is reduced by a factor proportional to $\delta/a$ and weighted by $|\Psi(a)|^2$ compared to the uniform case. This reflects the fact that the field intensity has already decreased to approximately 40\% of its peak value at the boundary.

\begin{figure}[h!]
  \centering
  \includegraphics[width=0.9\textwidth]{../figures/04_radius_validation.pdf}
  \caption{%
    \textbf{Perturbation theory validation: radius variations.}
    Left: Change in effective index $\Delta n_{\text{eff}}$ versus core radius variation $\Delta a$, comparing first-order perturbation theory (solid line) to exact numerical solution (dashed line with markers).
    The theory captures the essential physics across $\pm 0.1\,\mu\text{m}$ variations, with mean absolute error of 3.47\% and maximum error of 3.37\%, validating the first-order approximation.
    Right: Relative error between theory and exact, confirming systematic underprediction ($\approx 1.8\%$ mean) and demonstrating the validity envelope for perturbation theory.
  }
  \label{fig:radius_validation}
\end{figure}

\subsubsection{Radius Variation Sensitivity}

For a core radius perturbation $a \to a + \Delta a$, numerical differentiation of the dispersion relation yields:
\[
\frac{\partial n_{\text{eff}}}{\partial a} = \frac{n_{\text{eff}}(a + \Delta a) - n_{\text{eff}}(a)}{\Delta a}.
\]

For the reference fiber at $\lambda_0 = 1.55 \, \mu$m:
\begin{itemize}
  \item $\partial n_{\text{eff}}/\partial a \approx +1.2 \times 10^{-3} \, \mu\text{m}^{-1}$
  \item For $\Delta a = +0.1 \, \mu$m (2.5\% increase): $\Delta n_{\text{eff}} \approx +1.2 \times 10^{-4}$
  \item For $\Delta a = -0.1 \, \mu$m: $\Delta n_{\text{eff}} \approx -1.2 \times 10^{-4}$
\end{itemize}

The positive sign indicates that increasing the core radius increases confinement and raises $n_{\text{eff}}$. This sensitivity is weaker than the index sensitivity by approximately a factor of 3, reflecting the fact that radius variations affect the V-number (which enters through $u$ and $w$) rather than directly modifying the potential $k_0^2 n^2(r)$.

The radius sensitivity depends on wavelength through the V-number. At shorter wavelengths (higher V), the mode is more strongly confined and less sensitive to radius changes. At longer wavelengths approaching cutoff, the sensitivity diverges as $\partial n_{\text{eff}}/\partial a \propto (V^2)^{-1}$.

\subsubsection{Attenuation vs Conductivity}

For weak material absorption in the core with extinction coefficient $\kappa_0$, the attenuation coefficient is:
\[
\alpha = \frac{k_0 n_1 \kappa_0}{n_{\text{eff}}} P_{\text{core}}.
\]

Converting to dB/km using $\alpha_{\text{dB/km}} = 10 \log_{10}(e) \cdot \alpha \times 10^3 \approx 4.343 \times 10^3 \alpha$ (with $\alpha$ in m$^{-1}$):

For the reference fiber at $\lambda_0 = 1.55 \, \mu$m with $k_0 = 2\pi/\lambda_0 \approx 4.05 \times 10^6$ m$^{-1}$, $n_1 = 1.4504$, $n_{\text{eff}} = 1.447$, and $P_{\text{core}} = 0.77$:
\begin{itemize}
  \item For $\kappa_0 = 10^{-11}$: $\alpha \approx 3.04 \times 10^{-5} \, \text{m}^{-1} \approx 0.13 \, \text{dB/km}$
  \item For $\kappa_0 = 10^{-10}$: $\alpha \approx 3.04 \times 10^{-4} \, \text{m}^{-1} \approx 1.3 \, \text{dB/km}$
  \item For $\kappa_0 = 10^{-9}$: $\alpha \approx 3.04 \times 10^{-3} \, \text{m}^{-1} \approx 13 \, \text{dB/km}$
\end{itemize}

Ultra-low-loss fibers achieve attenuation below 0.2 dB/km at $\lambda_0 = 1.55$ $\mu$m, corresponding to $\kappa_0 \lesssim 1.5 \times 10^{-11}$. This demonstrates the extreme purity required in optical fiber manufacturing. Even trace impurities or defects introducing small imaginary index components can produce measurable loss.

The attenuation scales linearly with $\kappa_0$ and proportionally to the power confinement factor $P_{\text{core}}$. Fibers with weaker confinement (more power in the cladding) are less sensitive to core absorption but more sensitive to cladding defects or surface roughness.

\subsection{Perturbation Theory Accuracy}

To assess the validity range of first-order perturbation theory, we compare the perturbative prediction $\Delta n_{\text{eff}}^{\text{(pert)}}$ with the exact change $\Delta n_{\text{eff}}^{\text{(exact)}}$ obtained by re-solving the dispersion relation.

\subsubsection*{Index Perturbation Sensitivity}

For uniform core index perturbations:
\begin{itemize}
  \item $|\Delta n_1/n_1| < 0.01$ (1\%): Relative error $< 1\%$, excellent agreement
  \item $|\Delta n_1/n_1| \approx 0.02$ (2\%): Relative error $\approx 3$--$5\%$, acceptable for many applications
  \item $|\Delta n_1/n_1| > 0.05$ (5\%): Relative error $> 10\%$, second-order corrections needed
\end{itemize}

The accuracy degrades for larger perturbations because:
\begin{enumerate}
  \item The mode profile $\Psi(r)$ itself changes, violating the assumption that the unperturbed profile is valid
  \item The normalization changes nonlinearly with the perturbation magnitude
  \item Higher-order terms $(\Delta n^2)^2, (\Delta n^2)^3, \ldots$ become non-negligible
\end{enumerate}

\begin{figure}[h!]
  \centering
  \includegraphics[width=0.95\textwidth]{../figures/05_index_sensitivity.pdf}
  \caption{%
    \textbf{First-order perturbation sensitivity: index variations.}
    Contour map of $\Delta n_{\text{eff}}$ as a function of index perturbation magnitude $\Delta n$ and perturbation shell width $w_p$.
    The quadratic dependence on $\Delta n$ (as predicted by the perturbation integral $\propto (\Delta n)^2$) is clearly visible in the color gradient.
    The 400-point sensitivity map spans $\Delta n \in [10^{-5}, 10^{-2}]$ and $w_p \in [0.01, 1.0]\,\mu\text{m}$, encompassing practical doping, diffusion, and thermal index variations.
    Peak sensitivity occurs for perturbations near the fiber center where mode power density is highest.
  }
  \label{fig:index_sensitivity}
\end{figure}

\subsubsection*{Radius Perturbation}

For radius variations:
\begin{itemize}
  \item $|\Delta a/a| < 0.05$ (5\%): Relative error $< 2\%$
  \item $|\Delta a/a| \approx 0.10$ (10\%): Relative error $\approx 5$--$8\%$
  \item $|\Delta a/a| > 0.20$ (20\%): Relative error $> 15\%$
\end{itemize}

Radius perturbations are slightly less sensitive to higher-order effects than index perturbations because the geometric change affects the V-number rather than directly modifying the eigenvalue problem potential. However, for large $|\Delta a|$, the effective perturbation becomes spatially extended and the local approximations break down.

\subsubsection*{General Observations}

First-order perturbation theory is remarkably accurate for realistic fabrication tolerances (typically $\lesssim 1$--$2\%$ for index and $\lesssim 5\%$ for radius). This validates its use as a predictive tool for:
\begin{itemize}
  \item Process sensitivity analysis during fiber design
  \item Yield estimation in manufacturing
  \item Error budgeting in precision sensing applications
  \item Rapid parametric studies where exact recomputation would be computationally expensive
\end{itemize}

For perturbations exceeding the first-order validity range, iterative perturbation methods or direct numerical solution of the perturbed problem are recommended.
