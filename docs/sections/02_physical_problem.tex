% 02_physical_problem.tex
\section{Physical Problem}

\subsection{Step-Index Fiber Geometry}

We consider a cylindrical optical fiber with a step-index refractive index profile, consisting of a high-index core surrounded by a lower-index cladding. The fiber extends infinitely along the $z$-axis and exhibits azimuthal symmetry, allowing us to work in cylindrical coordinates $(r, \phi, z)$.

The geometric parameters are:
\begin{itemize}
  \item Core radius: $a$
  \item Radial coordinate: $r \in [0, \infty)$
  \item Fiber axis aligned with $\zhat$
\end{itemize}

The step-index structure represents an idealized model widely used in telecommunications (single-mode and multimode fibers) and serves as the foundation for more complex graded-index profiles. This geometry admits analytical solutions in terms of Bessel functions, making it an ideal testbed for perturbation analysis.

\begin{figure}[h!]
  \centering
  \includegraphics[width=0.85\textwidth]{../figures/01_fiber_schematic.pdf}
  \caption{%
    \textbf{Step-index fiber geometry and refractive index profile.}
    Left: Cross-sectional diagram of cylindrical fiber with core radius $a$, core index $n_1$, and cladding index $n_2$.
    Right: Radial refractive index profile showing step discontinuity at $r=a$.
    The V-number $V = k_0 a\sqrt{n_1^2-n_2^2}$ determines the number of guided modes; for $V < 2.405$, only the fundamental LP$_{01}$ mode propagates.
  }
  \label{fig:fiber_schematic}
\end{figure}

\subsection{Refractive Index Profile}

The refractive index distribution is piecewise constant:
\[
n(r) =
\begin{cases}
n_1, & r < a \quad \text{(core)}, \\
n_2, & r > a \quad \text{(cladding)},
\end{cases}
\]
where $n_1 > n_2$ to ensure wave confinement through total internal reflection.

The index contrast is characterized by:
\begin{itemize}
  \item Absolute difference: $\Delta n = n_1 - n_2$
  \item Relative difference: $\Delta = \frac{n_1^2 - n_2^2}{2n_1^2} \approx \frac{n_1 - n_2}{n_1}$ for $n_1 \approx n_2$
  \item Normalized frequency (V-number): $V = k_0 a \sqrt{n_1^2 - n_2^2}$
\end{itemize}

The V-number determines the number of guided modes: for $V < 2.405$, only the fundamental LP$_{01}$ mode propagates (single-mode regime). Typical telecommunications fibers operate near this cutoff with $V \approx 2.0$--$2.4$.

The free-space wavenumber is:
\[
k_0 = \frac{2\pi}{\lambda_0},
\]
where $\lambda_0$ is the vacuum wavelength.

\subsection{Scalar Helmholtz Equation}

Maxwell's equations in a source-free, isotropic dielectric reduce to the vector wave equation for the electric field $\vec{E}$:
\[
\nabla^2 \vec{E} + k_0^2 n^2(r) \vec{E} = 0.
\]

For propagating modes with $z$-dependence $\exp(-\imag\beta z)$ and $\phi$-dependence $\exp(\imag m\phi)$, each component of $\vec{E}$ satisfies a scalar equation in the radial coordinate. Under the weakly guiding approximation (discussed below), this reduces to the scalar Helmholtz equation for a single transverse field component $\Psi(r)$:
\[
\frac{1}{r}\dv{r}\left( r\dv{\Psi}{r} \right)
+ \left( k_0^2 n^2(r) - \beta^2 - \frac{m^2}{r^2} \right) \Psi(r) = 0,
\]
where $\beta$ is the propagation constant and $m$ is the azimuthal mode number.

This equation governs the radial profile of linearly polarized (LP) modes, which form an approximate basis for weakly guiding fibers.

\subsection{Weakly Guiding Approximation}

The weakly guiding approximation assumes that the refractive index contrast is small:
\[
\Delta = \frac{n_1 - n_2}{n_1} \ll 1.
\]

Typical values for telecommunications fibers are $\Delta \approx 0.003$--$0.01$ (0.3\%--1\%).

Under this approximation, several simplifications arise:

\begin{enumerate}
  \item \textbf{Scalar approximation:} The vectorial coupling between transverse electric and magnetic field components becomes negligible, allowing independent treatment of quasi-TE and quasi-TM polarizations. The LP mode description becomes accurate.
  
  \item \textbf{Degenerate modes:} Exact vector modes (HE, EH, TE, TM) become nearly degenerate in pairs, forming LP mode groups characterized by a single azimuthal number $m$ and radial number $l$.
  
  \item \textbf{Simplified dispersion relation:} The characteristic equation for the propagation constant simplifies significantly, involving only scalar Bessel functions rather than coupled systems.
  
  \item \textbf{Polarization independence:} To leading order in $\Delta$, the propagation constant is independent of polarization state.
\end{enumerate}

The validity of this approximation extends to most practical fiber configurations, except for highly birefringent fibers or large-core multimode designs. For the fundamental LP$_{01}$ mode, it provides excellent accuracy for $\Delta < 0.01$.

Within this framework, the electric field of an LP mode takes the approximate form:
\[
\vec{E}(r, \phi, z, t) = \Psi(r) \cos(m\phi) \, \vec{e}_{\text{pol}} \, \exp(-\imag\beta z + \imag\omega t),
\]
where $\vec{e}_{\text{pol}}$ is a fixed transverse polarization vector, and $\Psi(r)$ satisfies the scalar Helmholtz equation derived above.

For the fundamental mode, $m = 0$, yielding azimuthally symmetric solutions $\Psi(r)$ that depend only on the radial coordinate.
